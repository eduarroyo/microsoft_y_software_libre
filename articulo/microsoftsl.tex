\documentclass[10pt, titlepage]{article}
% preámbulo
\usepackage{lmodern}
\usepackage[T1]{fontenc}
\usepackage[document]{ragged2e}
\usepackage[spanish,activeacute]{babel}
\usepackage{hyperref}
\usepackage{geometry}
\usepackage{float}

\usepackage{xcolor}
\geometry{
    a4paper,
    left=25mm,
    right=25mm,
    top=25mm,
    bottom=25mm
}
\usepackage{fancyhdr}

\pagestyle{fancy}
\fancyhf{}

\lhead{\rightmark}
\rhead{Microsoft y Software Libre}
\lfoot{Eduardo Arroyo Ramírez}
\rfoot{Página \thepage}

\def\code#1{\texttt{#1}}

\hypersetup{
    colorlinks=true,
    linkcolor=blue,
    filecolor=blue,
    urlcolor=blue,
    citecolor=black
}

\title{Microsoft y Software Libre}
\author{Eduardo Arroyo Ramírez}

\begin{document}

%\graphicspath{ {figuras} }

\makeatletter
\begin{titlepage}
    \begin{center}
        {\scshape\Large Escuela Politécnica Superior \par}
        \vspace{0.5cm}
        {\scshape\large Universidad de Córdoba \par}
        \vspace{6cm}
        {\scshape\Huge Microsoft y Software Libre \par}
        \vspace{0.5cm}
        {\itshape\Large Software Libre y Compromiso Social \par}
    \end{center}
    \vspace{11cm}
    \begin{flushright}
        \@author\space \\
        i12arrae@uco.es \\
        Curso 2019/2020
    \end{flushright}
\end{titlepage}

\tableofcontents
\listoffigures
\listoftables
\clearpage

% Aquí comienza el cuerpo del documento
\justify
\section{Introducción}
Microsoft es una multinacional fundada el 4 de abril de 1975 por Bill Gates y Paul Allen en Albuquerque, Nuevo México. Desde sus inicios, la compañía se ha caracterizado por su cuestionable ética y su agresiva estrategia empresarial, utilizando su posición dominante para ``exterminar'' a la competencia e imponer sus productos y sus estándares. Además, la compañía de Redmond ha sido históricamente un enemiga acérrima del software libre, pero desde que Satya Nadella fuera nombrado CEO de Microsoft en 2014, la compañía ha dado un giro de 180º a su postura al respecto.

En este trabajo se repasa el origen y la motivación del cambio de paradigma de la compañía y las repercusiones que dicho cambio ha tenido. También se describirán los productos que Microsoft ha liberado, como .NET Framework, Visual Studio Code, Xamarin.Forms, y cómo ahora están soportados por plataformas como Linux o MacOS.

\section{Relación de Microsoft con el Software Libre}

\subsection{El estilo Microsoft: Embrace, extend and extinguish}

\subsubsection{OS/2 vs Windows 3.X}
\href{https://www.quora.com/Why-did-IBMs-OS-2-project-lose-to-Microsoft-given-that-IBM-had-much-more-resources-than-Microsoft-at-that-time}{Microsoft desarrolló OS/2} como socio de IBM y luego se desmarcó para desarrollar su Windows 3.X desde una posición de ventaja y hacer la competencia a OS/2.

\subsubsection{La guerra de los navegadores}
Durante los años 90 tuvo lugar la llamada \href{https://es.wikipedia.org/wiki/Guerra_de_navegadores}{Guerra de Navegadores} en la que Microsoft logró imponer su navegador Internet Explorer distribuyéndolo gratuitamente con Windows 95, lo que asfixió comercialmente a su principal competidor, Netscape Communicator, que contaba con un producto mejor. Aunque Microsoft fue condenado en un \href{https://es.wikipedia.org/wiki/Caso_Estados_Unidos_contra_Microsoft}{juicio por monopolio}, tras las apelaciones y el acuerdo final, las medidas adoptadas finalmente no obligaban a la compañía a realizar cambios efectivos en su estructura o su política comercial.

\subsubsection{Embrace, extend and extinguish}
El departamento de justicia de los estados unidos descubrió que Microsoft usaba la expresión ``\href{https://en.wikipedia.org/wiki/Embrace,_extend,_and_extinguish}{embrace, extend and extinguish}'' internamente para describir su estrategia para introducir sus productos en ecosistemas con estándares establecidos, extender dichos estándares con elementos patentados y luego utilizar las diferencias para perjudicar a sus competidores.

\subsection{Ataques al mundo del Software Libre}

\subsubsection{1998-2004: The Halloween Documents}
Se conoce como los ``\href{https://en.wikipedia.org/wiki/Halloween_documents}{Halloween Documents}'' una serie de documentos confidenciales de Microsoft sobre potenciales estrategias relacionadas con el software libre/open-source y sobre Linux en particular, y una serie de respuestas de los medios a estos documentos.

\subsubsection{2001: ``Linux is a cancer''}

\subsubsection{2010: Las guerras de patentes}

\subsubsection{2014: Balmer contra LiMux}
 \href{https://www.muycomputer.com/2014/05/16/linux-en-munich/}{Balmer contra el cambio a Linux en Munich}

\subsection{La etapa de Satya Nadella}
Inicio 4/2/2014
Cambio de paradigma: cloud computing, software como servicio: el software de servidor es territorio libre.

\subsubsection{2014: Microsoft abre .NET y lo lleva a Linux y OS}
\href{https://arstechnica.com/information-technology/2014/11/microsoft-open-sources-net-takes-it-to-linux-and-os-x/}{Microsoft abre .NET y lo lleva a Linux y OS}

\subsubsection{2016: Xamarin}
\href{https://www.genbeta.com/desarrollo/microsoft-adquiere-xamarin}{Compra de Xamarin}

\href{https://www.petri.com/microsofts-newly-acquired-xamarin-expands-developer-tools-new-features}{Hacen abierto el SDK de Xamarin}

\subsubsection{2018: Compra de Github}
\href{https://www.xataka.com/aplicaciones/oficial-microsoft-compra-github-7-500-millones-dolares}{Compra de GitHub 1}

\href{https://www.elconfidencial.com/tecnologia/2018-06-04/microsoft-ahora-ama-el-software-libre-compra-la-startup-github-por-7-500-millones_1573704/}{Compra de GitHub 2}

\href{https://www.xataka.com/aplicaciones/microsoft-punto-adquirir-github-desarrolladores-desarrolladores-desarrolladores}{Compra de GitHub 3}

\subsubsection{2018: Open Invention Network}
Hacer hincapié en las patentes que aporta, relación con la guerra de patentes en el pasado.
\href{https://azure.microsoft.com/en-us/blog/microsoft-joins-open-invention-network-to-help-protect-linux-and-open-source/}{Microsoft se une a la Open Invention Network 1}

\href{https://www.zdnet.com/article/what-does-microsoft-joining-the-open-invention-network-mean-for-you/}{Microsoft se une a la Open Invention Network 2}

\href{https://www.muycomputerpro.com/2018/10/10/microsoft-se-une-a-open-invention-network}{Microsoft se une a la Open Invention Network 3}

\subsubsection{2019: Charla de Richard Stallman en Microsoft}
\href{https://stallman.org/articles/microsoft-talk.html}{Charla de Stallman en Microsoft}

\subsubsection{2019: Petición de liberar Windows 7}
De llegar a liberar windows 7, quizás les interesaría utilizar una licencia copyleft para que nadie les hiciera competencia vendiendo su producto como no-libre.

\section{Licencias de Microsoft}
Cita: \cite{desde_linux_2020}

\subsection{Microsoft Public License (MS-PL)}
La \href{https://opensource.org/licenses/MS-PL}{Licencia Púbica de Microsoft (MS-PL)} \cite{microsoft_public_license_mspl} es la menos restrictiva de todas las licencias de la compañía. Permite la distribución y redistribución de código fuente tanto para fines comerciales como para no comerciales.

Esta licencia ha sido reconocida como licencia libre/abierta por la Free Software Foundation y por la Open Source Initiative \cite{desde_linux_2020}.

\subsection{Microsoft Reciprocal License (MS-RL)}
La \href{https://opensource.org/licenses/MS-RL}{Licencia Recíproca de Microsoft} permite la distribución de código derivado siempre y cuando los archivos fuentes estén incluidos y mantengan la licencia MS-RL.

Cita: \cite{microsoft_reciprocal_license_msrl}
\subsection{Licencias privativas, cerradas y comerciales}
\subsubsection{Microsoft Reference source License (MS-RSL)}
Esta es la más restrictiva de las licencias de código compartido de Microsoft. El código fuente está disponible sólo para verse con fines de referencia, principalmente para poder ver las clases de código fuente de Microsoft durante la depuración. Los desarrolladores no pueden distribuir o modificar el código para fines comerciales o no comerciales. La licencia ha sido anteriormente abreviada Ms-RL, pero Ms-RL ahora se refiere a la Licencia Recíproca de Microsoft.
\subsubsection{Microsoft Limited Public License (MS-LPL)}
Esta es una versión de la licencia pública de Microsoft en la que los derechos sólo se conceden a los desarrolladores de software basado en Microsoft Windows. Esta licencia no es de código abierto, tal como se define por la OSI, ya que viola la condición de que las licencias de código abierto debe ser tecnológicamente neutrales.
\subsubsection{Microsoft Limited Reciprocal License (MS-LRL)}
Esta es la versión de la Licencia Recíproca de Microsoft en la que los derechos sólo se conceden cuando se desarrolla software para una plataforma Microsoft Windows. Al igual que la Ms-LPL, esta licencia no es de código abierto porque no es tecnológicamente neutral.
\section{Productos de Software Libre}
\href{https://opensource.microsoft.com/explore}{Contribute to Microsoft open source projects}
\subsection{.NET Framework}
\subsection{.NET Core}
\subsection{Entity Framework Core}
\subsection{SignalR}
\subsection{Xamarin}
\subsection{Visual Studio Code}

\section{Aplicación .NET Core desarrollada en entorno Linux}
Descripción de un proyecto de ejemplo desarrollado en entorno Linux con .NET Core utilizando MVC y SignalR, con Visual Studio Code.

\section{Conclusiones}

\clearpage
\bibliographystyle{ieeetr}
\bibliography{microsoftsl}
\addcontentsline{toc}{section}{Bibliografía}

\end{document}