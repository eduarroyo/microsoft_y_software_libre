\documentclass[10pt, titlepage]{article}
% preámbulo
\usepackage{lmodern}
\usepackage[T1]{fontenc}
\usepackage[document]{ragged2e}
\usepackage[spanish,activeacute]{babel}
\usepackage{hyperref}
\usepackage{geometry}
\usepackage{float}

\usepackage{xcolor}
\geometry{
    a4paper,
    left=25mm,
    right=25mm,
    top=25mm,
    bottom=25mm
}
\usepackage{fancyhdr}

\pagestyle{fancy}
\fancyhf{}
\rhead{Eduardo Arroyo Ramírez \\ i12arrae@uco.es}
\lhead{Microsoft y Software Libre \\ Software Libre y Compromiso Social }
\lfoot{\rightmark}
\rfoot{Página \thepage}

\def\code#1{\texttt{#1}}

\hypersetup{
    colorlinks=true,
    linkcolor=blue,
    filecolor=blue,
    urlcolor=blue,
    citecolor=black
}

\title{Microsoft y Software Libre}
\author{Eduardo Arroyo Ramírez}

\begin{document}

%\graphicspath{ {figuras} }

\makeatletter
\begin{titlepage}
    \begin{center}
        {\scshape\Large Escuela Politécnica Superior \par}
        \vspace{0.5cm}
        {\scshape\large Universidad de Córdoba \par}
        \vspace{6cm}
        {\scshape\Huge Microsoft y Software Libre \par}
        \vspace{0.5cm}
        {\itshape\Large Software Libre y Compromiso Social \par}
    \end{center}
    \vspace{11cm}
    \begin{flushright}
        \@author\space \\
        i12arrae@uco.es \\
        Curso 2019/2020
    \end{flushright}
\end{titlepage}

\tableofcontents
\listoffigures
\listoftables
\clearpage

% Aquí comienza el cuerpo del documento
\justify
\section{Introducción}
Desde que Satya Nadella fuera nombrado CEO de Microsoft en 2014, la compañía ha dado un giro de 180º en su posición respecto al software libre. En este trabajo se repasa el origen y la motivación del cambio de paradigma de la compañía y las repercusiones que dicho cambio ha tenido. También se describirán los productos que Microsoft ha liberado como .NET Framework, Visual Studio Code, Xamarin.Forms, y cómo ahora están soportados por plataformas como Linux o MacOS.

\section{Relación de Microsoft con el Software Libre}
\subsection{La era Gates}
\subsection{La era Balmer}
\subsection{La era Nadella}
\section{Licencias de Microsoft}
Cita: \cite{desde_linux_2020}
\subsection{Microsoft Public License (MS-PL)}
Cita: \cite{microsoft_public_license_mspl}
\subsection{Microsoft Reciprocal License (MS-RL)}
Cita: \cite{microsoft_reciprocal_license_msrl}
\subsection{Licencias privativas, cerradas y comerciales}
\subsubsection{Microsoft Reference source License (MS-RSL)}
\subsubsection{Microsoft Limited Public License (MS-LPL)}
\subsubsection{Microsoft Limited Reciprocal License (MS-LRL)}
\section{Productos de Software Libre}
\subsection{.NET Framework}
\subsection{.NET Core}
\subsection{Entity Framework Core}
\subsection{SignalR}
\subsection{Xamarin}
\subsection{Visual Studio Code}
\section{Conclusiones}

\bibliographystyle{ieeetr}
\bibliography{microsoftsl}
\addcontentsline{toc}{section}{Bibliografía}

\end{document}