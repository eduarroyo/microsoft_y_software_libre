\section{El estilo Microsoft}
Microsoft es una multinacional fundada el 4 de abril de 1975 por Bill Gates y Paul Allen en Albuquerque, Nuevo México. Desde sus inicios, la compañía se ha caracterizado por su cuestionable ética y su agresiva estrategia empresarial, así como por el uso de su posición dominante para <<exterminar>> a la competencia e imponer sus productos y sus estándares. Prueba de ello son los \href{https://en.wikipedia.org/wiki/Microsoft_litigation}{numerosos juicios} en los que la compañía se ha visto envuelta\cite{wiki_2020:microsoft_litigation}. Con el fin de formar una imagen de la compañía, se resumen a continuación algunos hechos que tuvieron a Microsoft como protagonista.

\subsection{IBM y Microsoft}
A principios de los 80 IBM dominaba cómodamente el mercado de la computación para empresas. Con la llegada de los ordenadores personales, la compañía decidió contratar los servicios de Microsoft para el desarrollo de su sistema operativo para sus IBM-PC. Para ello, Microsoft compró los derechos no-exclusivos del sistema operativo QDOS, en el cuál basaría el desarrollo del nuevo sistema, a la empresa SCP. Microsoft desarrolló MS-DOS y lo licenció a IBM como PC-DOS. El único componente exclusivo de IBM era la BIOS.\cite{tdith_deal_with_devil_2019}\cite{tdith_rights_to_86DOS:2019}

Supuestamente, los requisitos de tiempo del desarrollo del nuevo sistema operativo forzaron a los desarrolladores a utilizar malas prácticas en el desarrollo que permitían a los desarrolladores de aplicaciones realizar accesos a los recursos hardware en lugar de hacer uso de llamadas estándar a funciones de la BIOS. Esto permitió a otros fabricantes de hardware clonar la arquitectura del IBM-PC rápidamente. En 1982, Compaq lanzó el Compaq Portable, el primer ordenador PC-Compatible. Microsoft inteligentemente había incluído en su contrato con IBM una cláusula que les permitía vender su sistema operativo MS-DOS. Todo ello encumbró a Microsoft como compañía líder en software de ordenadores personales, desplazando a IBM de su hegemonía total.\cite{wiki:ibm-pc-compatible-origins}

\subsection{La guerra de los navegadores}
Durante los años 90 tuvo lugar la conocida como <<\href{https://es.wikipedia.org/wiki/guerra_de_navegadores}{Guerra de Navegadores}>>\cite{wiki:guerra_navegadores_2020} en la que Microsoft logró imponer su producto Internet Explorer distribuyéndolo gratuitamente junto con Windows 95, lo que asfixió comercialmente a su principal competidor, Netscape Communicator, que contaba con un producto mejor. Aunque Microsoft fue condenado en un \href{https://es.wikipedia.org/wiki/Caso_Estados_Unidos_contra_Microsoft}{juicio por monopolio}\cite{wiki:juicio_microsoft_2020}, tras las apelaciones y el acuerdo final, las medidas adoptadas finalmente no obligaban a la compañía a realizar cambios efectivos en su estructura o su política comercial.\cite{thotw_2019:Browser_Wars}

Años después tuvo lugar un \href{https://en.wikipedia.org/wiki/Microsoft_Corp._v._Commission}{juicio similar en la Unión Europea}\cite{wiki_2020:microsoft_v_commision} por abuso de posición dominante en mercado a raíz de protestas de Novell y SUN Microsystems. Este juicio acabó obligando a Microsoft a que divulgara información sobre algunos de sus productos y liberara una versión de Windows sin Windows Media Player

\subsection{Embrace, extend and extinguish}
El departamento de justicia de los estados unidos descubrió que Microsoft usaba la expresión <<\href{https://en.wikipedia.org/wiki/Embrace,_extend,_and_extinguish}{embrace, extend and extinguish}>> internamente para describir su estrategia por la cuál Microsoft decide en un primer momento <<adoptar>> un estándar público, después lo <<mejora>> añadiendo extensiones incompatibles con el estándar original, y termina <<extinguiendo>> el estándar público al imponer sus propias extensiones propietarias por medio de su dominio del mercado.\cite{wiki_2020:embrace_extend_extinguish}

\label{par:FUD}
Otra estrategia de dudosa ética utilizada por Microsoft fue la llamada <<\href{https://en.wikipedia.org/wiki/Fear,_uncertainty,_and_doubt}{FUD}>> por sus siglas en inglés <<Fear, Uncertainty and Doubt>>\cite{wiki_2020:FUD}. En el primero de los \href{https://en.wikipedia.org/wiki/Halloween_documents}{Halloween Documents}\cite{wiki_2020:halloween_documents}, se revela que Microsoft utilizaba normalmente dicho método para desacreditar públicamente los productos de la competencia.