\section{El estilo Microsoft: Embrace, extend and extinguish}
Microsoft es una multinacional fundada el 4 de abril de 1975 por Bill Gates y Paul Allen en Albuquerque, Nuevo México. Desde sus inicios, la compañía se ha caracterizado por su cuestionable ética y su agresiva estrategia empresarial, así como por el uso de su posición dominante para <<exterminar>> a la competencia e imponer sus productos y sus estándares. Como ejemplo de ello, se exponen a continuación algunos episodios de la historia de la compañía.

\subsection{OS/2 vs Windows 3.X}
\href{https://www.quora.com/Why-did-IBMs-OS-2-project-lose-to-Microsoft-given-that-IBM-had-much-more-resources-than-Microsoft-at-that-time}{Microsoft desarrolló OS/2} como socio de IBM y luego se desmarcó para desarrollar su Windows 3.X desde una posición de ventaja y hacer la competencia a OS/2.

\subsection{1995-2001: La guerra de los navegadores}
Durante los años 90 tuvo lugar la llamada \href{https://es.wikipedia.org/wiki/Guerra_de_navegadores}{Guerra de Navegadores} en la que Microsoft logró imponer su navegador Internet Explorer distribuyéndolo gratuitamente con Windows 95, lo que asfixió comercialmente a su principal competidor, Netscape Communicator, que contaba con un producto mejor. Aunque Microsoft fue condenado en un \href{https://es.wikipedia.org/wiki/Caso_Estados_Unidos_contra_Microsoft}{juicio por monopolio}, tras las apelaciones y el acuerdo final, las medidas adoptadas finalmente no obligaban a la compañía a realizar cambios efectivos en su estructura o su política comercial.

\href{https://thehistoryoftheweb.com/browser-wars/}{The History of the Browser Wars: When Netscape Met Microsoft}

\subsection{2006: Embrace, extend and extinguish}
El departamento de justicia de los estados unidos descubrió que Microsoft usaba la expresión <<\href{https://en.wikipedia.org/wiki/Embrace,_extend,_and_extinguish}{embrace, extend and extinguish}>> internamente para describir su estrategia por la cuál Microsoft decide en un primer momento <<adoptar>> un estándar público, después lo <<mejora>> añadiendo extensiones incompatibles con el estándar original, y termina <<extinguiendo>> el estándar público al imponer sus propias extensiones propietarias por medio de su dominio del mercado. Según algunos críticos, esta estrategia tiende a reforzar el dominio de Microsoft e inhibe la competencia.