\section{Relación de Microsoft con el Software Libre}
\subsection{Ataques al mundo del Software Libre}

\subsubsection{1998-2004: The Halloween Documents}
Se conocen como los <<\href{https://en.wikipedia.org/wiki/Halloween_documents}{Halloween Documents}>> una serie de documentos confidenciales de Microsoft sobre potenciales estrategias relacionadas con el software libre/open-source y sobre Linux en particular, y una serie de respuestas de los medios a estos documentos.

Estos documentos fueron publicados por Eric S. Raymond el 30 de octubre de 1998 y Microsoft ha reconocido su autenticidad. Los dos primeros documentos reconocen que los programas libres/abiertos constituyen una amenaza significativa al dominio de Microsoft, reconocen que dichos programas son más competitivos que los suyos y sugieren medios para interferir en su desarrollo. Se trata de una información muy importante ya que contradice varias declaraciones publicas de la compañía.

\subsubsection{2001: <<Linux is a cancer>>}
\href{https://www.theregister.co.uk/2001/06/02/ballmer_linux_is_a_cancer/}{Ballmer: <<Linux is a cancer>>}.
\subsubsection{2009: Hostilidades contra Linux}
\href {https://arstechnica.com/information-technology/2009/04/linux-foundation-says-its-time-to-ditch-microsofts-fat/}{La Linux Foundation dice que  }

\subsubsection{2010: Las guerras de patentes}

\subsubsection{2014: Balmer contra LiMux}
 \href{https://www.muycomputer.com/2014/05/16/linux-en-munich/}{Balmer contra el cambio a Linux en Munich}

\subsection{La etapa de Satya Nadella}
Inicio 4/2/2014
Cambio de paradigma: cloud computing, software como servicio: el software de servidor es territorio libre.

\subsubsection{2014: Microsoft abre .NET y lo lleva a Linux y OS}
\href{https://arstechnica.com/information-technology/2014/11/microsoft-open-sources-net-takes-it-to-linux-and-os-x/}{Microsoft abre .NET y lo lleva a Linux y OS}

\subsubsection{2016: Xamarin}
\href{https://www.genbeta.com/desarrollo/microsoft-adquiere-xamarin}{Compra de Xamarin}

\href{https://www.petri.com/microsofts-newly-acquired-xamarin-expands-developer-tools-new-features}{Hacen abierto el SDK de Xamarin}

\subsubsection{2018: Compra de Github}
\href{https://www.xataka.com/aplicaciones/oficial-microsoft-compra-github-7-500-millones-dolares}{Compra de GitHub 1}

\href{https://www.elconfidencial.com/tecnologia/2018-06-04/microsoft-ahora-ama-el-software-libre-compra-la-startup-github-por-7-500-millones_1573704/}{Compra de GitHub 2}

\href{https://www.xataka.com/aplicaciones/microsoft-punto-adquirir-github-desarrolladores-desarrolladores-desarrolladores}{Compra de GitHub 3}

\subsubsection{2018: Open Invention Network}
Hacer hincapié en las patentes que aporta, relación con la guerra de patentes en el pasado.
\href{https://azure.microsoft.com/en-us/blog/microsoft-joins-open-invention-network-to-help-protect-linux-and-open-source/}{Microsoft se une a la Open Invention Network 1}

\href{https://www.zdnet.com/article/what-does-microsoft-joining-the-open-invention-network-mean-for-you/}{Microsoft se une a la Open Invention Network 2}

\href{https://www.muycomputerpro.com/2018/10/10/microsoft-se-une-a-open-invention-network}{Microsoft se une a la Open Invention Network 3}

\subsubsection{2019: Charla de Richard Stallman en Microsoft}
\href{https://stallman.org/articles/microsoft-talk.html}{Charla de Stallman en Microsoft}

\subsubsection{2019: Petición de liberar Windows 7}
De llegar a liberar windows 7, quizás les interesaría utilizar una licencia copyleft para que nadie les hiciera competencia vendiendo su producto como no-libre.