\section{Relación de Microsoft con el Software Libre}
Durante los 90 Microsoft estuvo en la cima del mercado del software para ordenadores personales gracias a las estrategias de marketing de la compañía pero al final de la década, la compañía comenzó a percibir el movimiento de Software Libre y Open Source como una amenaza a sus beneficios. La década siguiente estuvo marcada por los ataques constantes de Microsoft al Software Libre.\cite{wiki_2020:microsoft_and_open_source}

\subsection{El juego sucio}
\subsubsection{The Halloween Documents}
Se conoce como <<\href{https://en.wikipedia.org/wiki/Halloween_documents}{The Halloween Documents}>> una serie de documentos confidenciales de Microsoft sobre potenciales estrategias relacionadas con el software libre/open-source y sobre Linux en particular, y una serie de respuestas de diversos medios a estos documentos.

Los primeros fueron publicados por Eric S. Raymond el 30 de octubre de 1998 y Microsoft ha reconocido su autenticidad. Los dos primeros documentos reconocen que los programas libres/abiertos constituyen una amenaza significativa al dominio de Microsoft, que dichos programas son más competitivos que los suyos y sugieren medios para interferir en su desarrollo. Se trata de una información muy importante ya que contradice varias declaraciones publicas de la compañía. \cite{wiki_2020:halloween_documents}

\subsubsection{Las declaraciones de Steve Ballmer}
Después de Bill Gates, Steve Ballmer fue CEO de Microsoft de 2000 a 2014. Entre sus declaraciones públicas, Ballmer dejó auténticas joyas, algunas de ellas referentes al software libre, en las que se puede ver un claro ejemplo de la estrategia <<\href{https://en.wikipedia.org/wiki/Fear,_uncertainty,_and_doubt}{FUD}>> mencionada anteriormente (ver \ref{par:FUD}).

En 2000, Ballmer afirmó que <<\href{https://www.theregister.co.uk/2000/07/31/ms_ballmer_linux_is_communism/}{Linux tiene las características del comunismo} que la gente tanto ama. Es decir, que es gratis>> \cite{lea_2000:ballmer_linux_comunism}. Más tarde, en 2001 diría <<\href{https://www.theregister.co.uk/2001/06/02/ballmer_linux_is_a_cancer/}{Linux es un cáncer} que se pega, en sentido de la propiedad intelectual, a todo lo que toca>>. \cite{greene_2018:ballmer_linux_cancer}

En 2016, habiendo tomado ya Nadella las riendas de la compañía, Ballmer rectificó su opinión sobre GNU/Linux y reconoció que es un enemigo real de Windows. También escribió un email a Satya Nadella para felicitarlo por las recientes decisiones de la compañía respecto a las nuevas políticas de la compañía respecto al Software Libre.\cite{tung_2016:ballmer_linux_no_more_cancer}

\subsubsection{2014: Balmer contra LiMux}
En 2003, la ciudad de Munich estaba apunto de aprobar una migración de los sistemas del ayuntamiento (aproximadamente 15000 equipos) a Linux cuando Steve Ballmer interrumpió sus vacaciones y voló hasta la ciudad para reunirse con su alcalde, Christian Ude, con el fin de persuadirlo para detener la migración. Ballmer trató de convencer al alcande de que sería una mala decisión cambiar a sistemas de código abierto porque no era <<algo en lo que la administración pudiera confiar>>. La migración se completó antes de 2008, ahorrando a la ciudad 11 millones de euros y reduciendo su dependencia de estándares propietarios y fabricantes.\cite{munich_linux_migration}

\subsection{La etapa de Satya Nadella}
En 2014 Satya Nadella fue nombrado CEO de Microsoft. Desde entonces, la compañía comenzó a adoptar el Software Libre/Open Source en sus principales áreas de negocio. En contraste con la etapa de Ballmer, Nadella presentó una imagen que rezaba ``Microsoft \heart\ Linux''. 
Inicio 4/2/2014
Cambio de paradigma: cloud computing, software como servicio: el software de servidor es territorio libre.

\subsubsection{Azure}

\subsubsection{2014: Microsoft abre .NET y lo lleva a Linux y OS}
\href{https://arstechnica.com/information-technology/2014/11/microsoft-open-sources-net-takes-it-to-linux-and-os-x/}{Microsoft abre .NET y lo lleva a Linux y OS}

\subsubsection{2015: Microsoft se une a la Linux Foundation}
 \href{hhttps://arstechnica.com/information-technology/2016/11/microsoft-yes-microsoft-joins-the-linux-foundation/}{Microsoft ---yes, Microsoft--- joins the Linux Foundation}

\subsubsection{2016: Windows Subsystem for Linux}


\subsubsection{2016: Xamarin}
\href{https://www.genbeta.com/desarrollo/microsoft-adquiere-xamarin}{Compra de Xamarin}

\href{https://www.petri.com/microsofts-newly-acquired-xamarin-expands-developer-tools-new-features}{Hacen abierto el SDK de Xamarin}

\subsubsection{2018: Compra de Github}
\href{https://www.xataka.com/aplicaciones/oficial-microsoft-compra-github-7-500-millones-dolares}{Compra de GitHub 1}

\href{https://www.elconfidencial.com/tecnologia/2018-06-04/microsoft-ahora-ama-el-software-libre-compra-la-startup-github-por-7-500-millones_1573704/}{Compra de GitHub 2}

\href{https://www.xataka.com/aplicaciones/microsoft-punto-adquirir-github-desarrolladores-desarrolladores-desarrolladores}{Compra de GitHub 3}

\subsubsection{2018: Open Invention Network}
Hacer hincapié en las patentes que aporta, relación con la guerra de patentes en el pasado.
\href{https://azure.microsoft.com/en-us/blog/microsoft-joins-open-invention-network-to-help-protect-linux-and-open-source/}{Microsoft se une a la Open Invention Network 1}

\href{https://www.zdnet.com/article/what-does-microsoft-joining-the-open-invention-network-mean-for-you/}{Microsoft se une a la Open Invention Network 2}

\href{https://www.muycomputerpro.com/2018/10/10/microsoft-se-une-a-open-invention-network}{Microsoft se une a la Open Invention Network 3}

\subsubsection{2019: Charla de Richard Stallman en Microsoft}
\href{https://stallman.org/articles/microsoft-talk.html}{Charla de Stallman en Microsoft}

\subsubsection{2019: Petición de liberar Windows 7}
De llegar a liberar windows 7, quizás les interesaría utilizar una licencia copyleft para que nadie les hiciera competencia vendiendo su producto como no-libre.

\subsection{Motivos para el cambio}
En la década de los 2000, Microsoft se encontraba cómodamente en la cima del mercado del software para ordenadores personales, pero la expansión del Smartphone en la década de 2010 junto con la aparición del cloud computing, forzaron a la compañía de Redmond a replantear su estrategia.

Microsoft tenía copado el mercado de escritorio con Windows, pero al descender las ventas de PC y aumentar el uso de software como servicio tuvieron que cambiar su estrategia. 