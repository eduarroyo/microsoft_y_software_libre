\section{Licencias de Microsoft}
Cita: \cite{desde_linux_2020}

\subsection{Microsoft Public License (MS-PL)}
La \href{https://opensource.org/licenses/MS-PL}{Licencia Púbica de Microsoft (MS-PL)} es la menos restrictiva de las licencias de Microsoft y permite la distribución de código compilado ya sea para fines comerciales como no comerciales bajo cualquier licencia que cumpla con la MS-PL. La redistribución del código fuente en sí únicamente se autoriza bajo la MS-PL. Inicialmente titulada Microsoft Permissive License, fue renombrada a Microsoft Public License, mientras que se estaba revisando para su aprobación por la Open Source Initiative (OSI). La licencia fue aprobada en 2007 junto con el MS-RL. De acuerdo con la Free Software Foundation, es una licencia de software libre. Sin embargo, no es compatible con la GNU GPL.\cite{wiki_2019:microsoft_public_license_mspl}

\subsection{Microsoft Reciprocal License (MS-RL)}
La \href{https://opensource.org/licenses/MS-RL}{Licencia Recíproca de Microsoft} permite la distribución de código derivado siempre y cuando los archivos fuentes estén incluidos y mantengan la licencia MS-RL. La MS-RL permite que aquellos archivos en la distribución que no contengan código originalmente licenciado bajo la MS-RL sean licenciados de acuerdo a la elección del titular de los derechos de autor. Esto es equivalente a la CDDL, la EPL o la LGPL. En un principio conocida como la Licencia Comunitaria de Microsoft, fue renombrada en el proceso de aprobación de OSI. En 2007 se anunció que Microsoft había presentado oficialmente la Ms-PL y la Ms-RL a OSI para su aprobación.  De acuerdo con la Free Software Foundation, es una licencia de software libre. Sin embargo, no es compatible con la GNU GPL.\cite{wiki_2019:microsoft_reciprocal_license_msrl}

\subsection{Licencias privativas, cerradas y comerciales}
\subsubsection{Microsoft Reference source License (MS-RSL)}
Esta es la más restrictiva de las licencias de código compartido de Microsoft. El código fuente está disponible sólo para verse con fines de referencia, principalmente para poder ver las clases de código fuente de Microsoft durante la depuración. Los desarrolladores no pueden distribuir o modificar el código para fines comerciales o no comerciales. La licencia ha sido anteriormente abreviada Ms-RL, pero Ms-RL ahora se refiere a la Licencia Recíproca de Microsoft.\cite{wiki_2019:microsoft_reference_source_license_msrsl}
\subsubsection{Microsoft Limited Public License (MS-LPL)}
Esta es una versión de la licencia pública de Microsoft en la que los derechos sólo se conceden a los desarrolladores de software basado en Microsoft Windows. Esta licencia no es de código abierto, tal como se define por la OSI, ya que viola la condición de que las licencias de código abierto debe ser tecnológicamente neutrales.\cite{wiki_2019:microsoft_limited_public_license_mslpl}
\subsubsection{Microsoft Limited Reciprocal License (MS-LRL)}
Esta es la versión de la Licencia Recíproca de Microsoft en la que los derechos sólo se conceden cuando se desarrolla software para una plataforma Microsoft Windows. Al igual que la Ms-LPL, esta licencia no es de código abierto porque no es tecnológicamente neutral.\cite{wiki_2019:microsoft_limited_reciprocal_license_mslrl}