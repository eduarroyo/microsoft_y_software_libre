\section{Licencias de Microsoft}
Cita: \cite{desde_linux_2020}

\subsection{Microsoft Public License (MS-PL)}
La \href{https://opensource.org/licenses/MS-PL}{Licencia Púbica de Microsoft (MS-PL)} \cite{microsoft_public_license_mspl} es la menos restrictiva de todas las licencias de la compañía. Permite la distribución y redistribución de código fuente tanto para fines comerciales como para no comerciales.

Esta licencia ha sido reconocida como licencia libre/abierta por la Free Software Foundation y por la Open Source Initiative \cite{desde_linux_2020}.

\subsection{Microsoft Reciprocal License (MS-RL)}
La \href{https://opensource.org/licenses/MS-RL}{Licencia Recíproca de Microsoft} permite la distribución de código derivado siempre y cuando los archivos fuentes estén incluidos y mantengan la licencia MS-RL.
Cita: \cite{microsoft_reciprocal_license_msrl}
\subsection{Licencias privativas, cerradas y comerciales}
\subsubsection{Microsoft Reference source License (MS-RSL)}
Esta es la más restrictiva de las licencias de código compartido de Microsoft. El código fuente está disponible sólo para verse con fines de referencia, principalmente para poder ver las clases de código fuente de Microsoft durante la depuración. Los desarrolladores no pueden distribuir o modificar el código para fines comerciales o no comerciales. La licencia ha sido anteriormente abreviada Ms-RL, pero Ms-RL ahora se refiere a la Licencia Recíproca de Microsoft.
\subsubsection{Microsoft Limited Public License (MS-LPL)}
Esta es una versión de la licencia pública de Microsoft en la que los derechos sólo se conceden a los desarrolladores de software basado en Microsoft Windows. Esta licencia no es de código abierto, tal como se define por la OSI, ya que viola la condición de que las licencias de código abierto debe ser tecnológicamente neutrales.
\subsubsection{Microsoft Limited Reciprocal License (MS-LRL)}
Esta es la versión de la Licencia Recíproca de Microsoft en la que los derechos sólo se conceden cuando se desarrolla software para una plataforma Microsoft Windows. Al igual que la Ms-LPL, esta licencia no es de código abierto porque no es tecnológicamente neutral.