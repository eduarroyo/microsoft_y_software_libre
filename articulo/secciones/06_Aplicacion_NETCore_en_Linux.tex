\section{Aplicación .NET Core desarrollada en entorno Linux}
Se ha desarrollado una pequeña aplicación con .NET Core sobre Linux con el fin de demostrar el uso de tecnologías libres de Microsoft en entornos Linux. Esta aplicación está disponible, junto con este documento y las transparencias en un \href{https://github.com/eduarroyo/microsoft_y_software_libre}{repositorio de Github}\footnote{https://github.com/eduarroyo/microsoft\_y\_software\_libre}.

Se trata de una aplicación web que utiliza la biblioteca de comunicaciones en tiempo real SignalR y el ORM Entity Framework Core para persistir algunos datos en una base de datos administrada por un servidor SQL-Server en el mismo sistema Linux. La aplicación tiene las siguientes funcionalidades:

\begin{itemize}
    \item Lista de tareas colaborativa con persistencia.
    \item Chat en tiempo real con persistencia de los mensajes.
    \item Pizarra colaborativa en tiempo real.
\end{itemize}

Para el desarrollo de la aplicación se han utilizado las siguientes herramientas:
\begin{itemize}
    \item sqlcmd: Herramienta de comandos para configuración de SQL-Server.
    \item Editor Visual Studio Code para la edición del código, la elaboración de este documento, de las transparencias y de las preguntas de test en formato GIFT.
    \item Plugin de Visual Studio Core para consultas contra la base de datos.
    \item Herramienta cliente de .NET Core (dotnet) para la gestión de paquetes del proyecto y la generación del modelo OO a partir del esquema relacional de la base de datos.
    \item Servidor web Kestrel.
\end{itemize}

La instalación de .net Core SDK se ha realizado siguiendo las \href{https://docs.microsoft.com/es-es/dotnet/core/install/linux-package-manager-ubuntu-1904}{instrucciones del Microsoft}\footnote{https://docs.microsoft.com/es-es/dotnet/core/install/linux-package-manager-ubuntu-1904}. La creación del proyecto ha partido de la plantilla de aplicación web MVC de .NET Core que se obtiene mediante la herramienta \code{dotnet} desde línea de comandos:

\begin{lstlisting}[basicstyle=\small,xrightmargin=.10\textwidth,xleftmargin=.10\textwidth,language=bash]
    $ dotnet new mvc -o tasklist
\end{lstlisting}

Utilizando la misma herramienta se han instalado la herramienta de scaffolding de Entity Framework Core y las dependencias del proyecto: SignalR y Entity Framework Core:

\begin{lstlisting}[basicstyle=\small,xrightmargin=.10\textwidth,xleftmargin=.10\textwidth,language=bash]
    $ dotnet tool install --global dotnet-ef
    $ dotnet add package Microsoft.AspNetCore.SignalR.Client
    $ dotnet add package Microsoft.EntityFrameworkCore.Design
    $ dotnet add package Microsoft.EntityFrameworkCore.SqlServer
\end{lstlisting}

El modelado de la base de datos se ha realizado mediante el plugin mssql de Visual Studio Code, que permite visualizar la estructura y realizar consultas de todo tipo contra una base de datos de SQL-Server. La clase de contexto de la base de datos, así como los modelos correspondientes a las tablas se generan automáticamente utilizando la herramienta de línea de comandos:

\begin{lstlisting}[basicstyle=\small,xrightmargin=.10\textwidth,xleftmargin=.10\textwidth,language=bash]
    $ dotnet ef dbcontext scaffold "<connection-string>" _ 
        Microsoft.EntityFrameworkCore.SqlServer -o Models
\end{lstlisting}