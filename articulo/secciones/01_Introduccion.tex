\section{Introducción}
Microsoft ha sido históricamente enemiga acérrima del software libre, pero desde que Satya Nadella fuera nombrado CEO de Microsoft en 2014, la compañía ha dado un giro de 180º a su postura al respecto liberando software, cediendo patentes y entrando en el negocio del software libre con Azure.

Este trabajo repasa el origen y la motivación del cambio de paradigma de la compañía y las repercusiones que dicho cambio ha tenido. También describe algunos de los productos que Microsoft ha liberado, como .NET Framework, Visual Studio Code o Xamarin.Forms y cómo ahora están soportados por plataformas como Linux o MacOS. Por último presenta un pequeño programa que ha sido desarrollado con tecnologías libres de Microsoft y con SQL Server utilizando GNU/Linux.