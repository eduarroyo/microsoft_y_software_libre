\section{Productos de Software Libre de Microsoft}Microsoft ha desarrollado numerosos proyectos Open Source que se pueden consultar en \href{https://opensource.microsoft.com}{Microsoft Open Source}. A continuación se exponen algunos de los proyectos OSS de Microsoft más significativos:

\subsection{.NET Core}
.NET Core es una plataforma de desarrollo de código abierto para uso general. Se pueden crear aplicaciones de .NET Core para Windows, macOS y Linux para procesadores x64, x86, ARM32 y ARM64 mediante los lenguajes C\#, Visual Basic y F\#. Se proporcionan marcos y API para la nube, IoT, la Interfaz de usuario de cliente y el aprendizaje automático.

La primera versión 1.0 apareció en junio de 2016, .NET Core 2.0 en agosto de 2017 y la última versión mayor, .NET Core 3.0, en mayo de 2019. Esta versión incluye soporte para aplicaciones de escritorio (sólo en entorno Windows), inteligencia artificial, aprendizaje automático e IoT.

.NET Core es un proyecto Open Source de la \href{https://dotnetfoundation.org/}{.NET Foundation}. Se puede participar registrando incidencias y preguntas en la \href{https://developercommunity.visualstudio.com/spaces/61/index.html}{comunidad de desarrolladores} y también aportando código en los \href{https://github.com/dotnet/core/blob/master/Documentation/core-repos.md}{repositorios en GitHub}.

\subsubsection{Características}
\begin{itemize}
    \item Multiplataforma: se ejecuta en los sistemas operativos Windows, macOS y Linux.
    \item Código abierto: el marco .NET Core es de código abierto, con licencias de MIT y Apache 2. .NET Core es un proyecto de .NET Foundation.
    \item Moderno: implementa paradigmas modernos como programación asincrónica, patrones que no son de copia que usan estructuras y gobernanza de recursos para contenedores.
    \item Rendimiento: proporciona alto rendimiento con características como intrínsecos de hardware, compilación en niveles e Intervalo<T>.
    \item Coherente entre entornos: el código se ejecuta con el mismo comportamiento en varios sistemas operativos y varias arquitecturas, como x64, x86 y ARM.
    \item Herramientas de línea de comandos: incluye herramientas de línea de comandos sencillas que se pueden usar para el desarrollo local y la integración continua.
    \item Implementación flexible: se puede incluir .NET Core en la aplicación o de forma paralela (instalaciones a nivel de usuario o de sistema). Se puede usar con contenedores de Docker.
\end{itemize}

\subsubsection{API}
.NET Core expone marcos para compilar cualquier tipo de aplicación:
\begin{itemize}
\item Aplicaciones en la nube con ASP.NET Core
\item Aplicaciones móviles con Xamarin
\item Aplicaciones de IoT con System.Device.GPIO
\item Aplicaciones cliente de Windows con WPF y Windows Forms
\item Aprendizaje automático ML.NET
\end{itemize}

.NET Core proporciona compatibilidad con las API .NET Framework y Mono API implementando la especificación de .NET Standard.

.NET Core consta de las siguientes partes:
\begin{itemize}
    \item El runtime de .NET Core, que proporciona un sistema de tipos, la carga de ensamblados, un colector de elementos no usados, interoperabilidad nativa y otros servicios básicos. Las bibliotecas de .NET Core Framework proporcionan tipos de datos primitivos, tipos de composición de aplicaciones y utilidades fundamentales.
    \item El runtime de ASP.NET, el cual proporciona un marco para crear aplicaciones modernas conectadas a Internet y basadas en la nube, como aplicaciones web, aplicaciones de IoT y back-ends móviles.
    \item El SDK de .NET Core y los compiladores de lenguaje (Roslyn y F\#) que habilitan la experiencia de desarrollador de .NET Core.
    \item El comando dotnet, que se usa para iniciar aplicaciones .NET Core y comandos de CLI. Se selecciona y lo hospeda, proporciona una directiva de carga de ensamblados e inicia aplicaciones y herramientas.
\end{itemize}

\subsubsection{Casos de éxito}
.NET Core es utilizado por numerosas compañías en productos de primer orden como \href{https://dotnet.microsoft.com/platform/customers}{GoDaddy}, \href{https://customers.microsoft.com/en-us/story/744483-setpoint-medical-discrete-manufacturing-net}{Setpoint Medical} o incluso \href{https://www.youtube.com/watch?v=1DIDWWKk8Bg&feature=youtu.be}{Stack Overflow}. Se pueden consultar más productos que utilizan la plataforma .NET Core en el \href{https://dotnet.microsoft.com/platform/customers}{.NET Customers Showcase}.

\subsection{Visual Studio Code}
Visual Studio Code es un editor de código fuente desarrollado por Microsoft para Windows , Linux y macOS. Incluye soporte para la depuración, control integrado de Git, resaltado de sintaxis, finalización inteligente de código, fragmentos y refactorización de código. También es personalizable, por lo que los usuarios pueden cambiar el tema del editor, los atajos de teclado y las preferencias. Es gratuito y de código abierto, aunque la descarga oficial está bajo software privativo e incluye características personalizadas por Microsoft.

Visual Studio Code se basa en Electron, un framework que se utiliza para implementar Chromium y Node.js como aplicaciones para escritorio, que se ejecuta en el motor de diseño Blink. Aunque utiliza el framework Electron, el software no usa Atom y en su lugar emplea el mismo componente editor (Monaco) utilizado en Visual Studio Team Services (anteriormente llamado Visual Studio Online).

\subsubsection{Características}
Visual Studio Code es compatible con varios lenguajes de programación. Muchas de las características de Visual Studio Code no están expuestas a través de los menús o la interfaz de usuario. Más bien, se accede a través de la paleta de comandos o a través de archivos .json (por ejemplo, preferencias del usuario). La paleta de comandos es una interfaz de línea de comandos.

Visual Studio Code se puede extender a través de complementos, disponible a través de un repositorio central. Esto incluye adiciones al editor y soporte de idiomas. Una característica notable es la capacidad de crear extensiones que analizan código, como linters y herramientas para análisis estático, utilizando el Protocolo de Servidor de Idioma.

\subsection{Xamarin}
\href{https://dotnet.microsoft.com/apps/xamarin}{Xamarin} es una plataforma de código abierto bajo licencia MIT para el desarrollo de  aplicaciones para iOS, Android y Windows con .NET. Xamarin es una capa de abstracción que administra la comunicación de código compartido con el código de plataforma subyacente. Xamarin se ejecuta en un entorno administrado que proporciona ventajas como la asignación de memoria y la recolección de elementos no utilizados. El proyecto está disponible bajo licencia MIT en su repositorio de \href{https://github.com/xamarin}{Github}.

Xamarin permite a los desarrolladores compartir un promedio del 90\% de la aplicación entre plataformas. Este patrón permite a los desarrolladores escribir toda la lógica de negocios en un solo lenguaje (o reutilizar el código de aplicación existente), pero conseguir un rendimiento y una apariencia nativos en cada plataforma.

Las aplicaciones de Xamarin se pueden escribir en PC o Mac, y compilar en paquetes de aplicación nativos, como un archivo .apk en Android o .ipa en iOS.

\subsection{Entity Framework Core}
Entity Framework 6 (EF6) es un asignador relacional de objetos (O/RM) probado para .NET con más de diez años de desarrollo de características y estabilización, aunque ya no se desarrolla activamente.

EF Core es una versión más moderna, ligera y extensible de Entity Framework que tiene capacidades y ventajas muy similares a EF6. EF Core es producto de una reescritura completa y contiene muchas características nuevas que no están disponibles en EF6, aunque todavía carece de algunas de las funcionalidades más avanzadas de asignación de EF6. Microsoft recomienda el uso de Entity Framework Core siempre y cuando las características del desarrollo se ajusten a los requisitos.

Ambos funcionan con SQL Server o SQL Azure, SQLite, Azure Cosmos DB, MySQL, PostgreSQL y muchas otras bases de datos a través de un modelo de complemento de proveedor de bases de datos.

Tanto EF6 como EF Core se distribuyen bajo licencia Apache 2.0 y sus fuentes están disponibles en sus respectivos \href{https://github.com/dotnet/ef6}{repositorios} en \href{https://github.com/dotnet/efcore}{Github}.

\subsection{.NET Core SignalR}
.NET Core SignalR es una biblioteca de Open Source distribuida junto con .NET Core bajo licencia Apache 2.0 que simplifica la incorporación de funcionalidades Web en tiempo real a las aplicaciones. La funcionalidad web en tiempo real permite que el código del lado servidor inserte contenido en los clientes al momento. SignalR proporciona una API para crear llamadas a procedimiento remoto (RPC)de servidor a cliente. Las RPC llaman a funciones de JavaScript en los clientes desde el código de .NET Core del lado servidor. El código fuente de esta biblioteca puede encontrarse en su \href{https://github.com/dotnet/AspNetCore/tree/master/src/SignalR}{repositorio de Github}.

Estas son algunas características de SignalR para ASP.NET Core:
\begin{itemize}
    \item Controla la administración de conexiones automáticamente.
    \item Envía mensajes a todos los clientes conectados simultáneamente. Por ejemplo, un salón de chat.
    \item Envía mensajes a clientes o grupos de clientes específicos.
    \item Escala para controlar el aumento del tráfico.
\end{itemize}

Existe una versión de SignalR para .NET Framework Standard que también se distribuye bajo Apache 2.0 y sus fuentes está disponibles en \href{https://github.com/SignalR/SignalR}{Github}.