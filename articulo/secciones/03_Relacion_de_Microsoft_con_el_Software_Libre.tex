\section{Relación de Microsoft con el Software Libre}
Durante los 90 Microsoft estuvo en la cima del mercado del software para ordenadores personales gracias a las estrategias de marketing de la compañía pero al final de la década, la compañía comenzó a percibir el movimiento de Software Libre y Open Source como una amenaza a sus beneficios. La década siguiente estuvo marcada por los ataques constantes de Microsoft al Software Libre.\cite{wiki_2020:microsoft_and_open_source}

\subsection{El juego sucio}
\subsubsection{The Halloween Documents}
Se conoce como <<\href{https://en.wikipedia.org/wiki/Halloween_documents}{The Halloween Documents}>> una serie de documentos confidenciales de Microsoft sobre potenciales estrategias relacionadas con el software libre/open-source y sobre Linux en particular, y una serie de respuestas de diversos medios a estos documentos.

Los primeros fueron publicados por Eric S. Raymond el 30 de octubre de 1998 y Microsoft ha reconocido su autenticidad. Los dos primeros documentos reconocen que los programas libres/abiertos constituyen una amenaza significativa al dominio de Microsoft, que dichos programas son más competitivos que los suyos y sugieren medios para interferir en su desarrollo. Se trata de una información muy importante ya que contradice varias declaraciones publicas de la compañía. \cite{wiki_2020:halloween_documents}

\begin{quote}
    Linux ... is trusted in mission critical applications, and – due to its open source code – has a long term credibility which exceeds many other competitive OSs
\end{quote}

\begin{quote}
    An advanced Win32 GUI user would have a short learning cycle to become productive [under Linux]
\end{quote}

\begin{quote}
    Long term, my simple experiments do indicate that Linux has a chance at the desktop market ...
\end{quote}

\begin{quote}
    Overall respondents felt the most compelling reason to support OSS was that it `Offers a low total cost of ownership (TCO)'
\end{quote}

\subsubsection{SCO vs IBM}
En 2003, el grupo SCO (antes Caldera International) afirmó ser el dueño de Unix y de otros sistemas operativos, y que Linux y otras variantes de Unix estaban violando su propiedad intelectual porque usaban código de Unix sin licencia. Primero SCO intentó estructurar un proceso directamente contra los usuarios finales y las compañías proveedoras, lo que hubiera extendido en el mercado el miedo a usar estos sistemas. La respuesta fueron varias demandas de copyright contra SCO, opr ejemplo por vulnerar la GPL. Al final, SCO emprendió sólo unos cuantos procesos legales contra IBM, Novell, DaimlerChrysler y Autozone. A su vez, RedHat demandó a SCO por realizar falsas afirmaciones que afectaban a sus negocios. \cite{wiki_2019:sco_linux}

Esta campaña de SCO fue sustancialmente financiada y promovida por Microsoft y por negocios fondos de inversión fuertemente vinculados con Redmond. En particular, un email interno filtrado de SCO indicaba que Microsoft había invertido 106 millones de dólares a través de BayStar y otros medios. El documento fue reconocido como verídico por Blake Stowell de SCO. BayStar afirmó que el acuerdo fue propuesto por Microsoft, pero que el dinero vino de otras fuentes. Además, en mayo de 2003, Microsoft pagó a SCO 6 millones de dólares por una licencia de <<Unix and Unix-related patents>>, a pesar de que SCO no poseía ninguna patente así. Este acuerdo fue reflejado en la prensa como apoyo financiero de Microsoft a SCO en su disputa legal contra IBM por, supuestamente, violar la licencia de Unix en el desarrollo de Linux.\cite{wiki_2019:sco_linux_microsoft_controverse}

En 2016, la mayoría de los casos habían sido resueltos y ninguna sentencia fue favorable a SCO.\cite{wiki_2019:sco_linux}

\subsubsection{Las declaraciones de Steve Ballmer}
Después de Bill Gates, Steve Ballmer fue CEO de Microsoft de 2000 a 2014. Entre sus declaraciones públicas, Ballmer dejó auténticas joyas, algunas de ellas referentes al software libre, en las que se puede ver un claro ejemplo de la estrategia <<\href{https://en.wikipedia.org/wiki/Fear,_uncertainty,_and_doubt}{FUD}>> mencionada anteriormente (ver \ref{par:FUD}).

En 2000, Ballmer afirmó que <<\href{https://www.theregister.co.uk/2000/07/31/ms_ballmer_linux_is_communism/}{Linux tiene las características del comunismo} que la gente tanto ama. Es decir, que es gratis>> \cite{lea_2000:ballmer_linux_comunism}. Más tarde, en 2001 diría <<\href{https://www.theregister.co.uk/2001/06/02/ballmer_linux_is_a_cancer/}{Linux es un cáncer} que se pega, en sentido de la propiedad intelectual, a todo lo que toca>>. \cite{greene_2018:ballmer_linux_cancer}

En 2016, habiendo tomado ya Satya Nadella las riendas de la compañía, Ballmer rectificó su opinión sobre GNU/Linux y reconoció que es un enemigo real de Windows. También escribió un email a Nadella para felicitarlo por las recientes decisiones de la compañía respecto a las nuevas políticas de la compañía respecto al Software Libre.\cite{tung_2016:ballmer_linux_no_more_cancer}

\subsubsection{2014: Balmer contra LiMux}
En 2003, la ciudad de Munich estaba apunto de aprobar una migración de los sistemas del ayuntamiento (aproximadamente 15000 equipos) a Linux cuando Steve Ballmer interrumpió sus vacaciones y voló hasta la ciudad para reunirse con su alcalde, Christian Ude, con el fin de persuadirlo para detener la migración. Ballmer trató de convencer al alcalde de que sería una mala decisión cambiar a sistemas de código abierto porque no era <<algo en lo que la administración pudiera confiar>>. La migración se completó antes de 2008, ahorrando a la ciudad 11 millones de euros y reduciendo su dependencia de estándares propietarios y fabricantes.\cite{munich_linux_migration}

\subsection{La apertura de Satya Nadella}
En 2014 Satya Nadella fue nombrado CEO de Microsoft. Desde su llegada, la compañía comenzó a adoptar el Software Libre/Open Source en sus principales áreas de negocio. En contraste con la etapa de Ballmer, Nadella presentó una imagen que rezaba ``Microsoft \heart\ Linux''. Estos cambios se han estado materializando desde entonces en forma de liberaciones de software, apertura de patentes, participación en fundaciones ligadas al Software Libre como nada menos que la Linux Foundation, Apache Foundation o Eclipse Foundation.

\subsubsection{2014: .NET Foundation, Roslyn y Microsoft .NET Core}
En marzo de 2014, Microsoft anunció la creación de .NET Foundation para promover el software libre en sus tecnologías. La primera contribución fue el compilador Roslyn, bajo licencia Apache. Gracias a ello, el compilador pudo ser adaptado a cualquier plataforma sin esperar a que lo hiciera la propia compañía. Esto supuso un gran paso hacia la universalización de las herramientas de desarrollo de Microsoft.\cite{xatakaw_2014:net_foundation}

En noviembre de 2014, Microsoft liberó .NET Core bajo licencia MIT y parte de .Net Framework. Con este lanzamiento, la compañía consiguió aumentar el interés y la colaboración de la comunidad en estos productos.\cite{arstechnica_2014:microsoft_open_sources_.NET}

\subsubsection{2016: Linux Foundation, WSL y Xamarin}
Tras toda una vida de ataques a Linux, de maniobras FUD y de financiar demandas legales para perjudicar a Linux, en 2016 y tras haberse convertido Microsoft en el quinto mayor contribuyente al Kernel 3.0 de Linux\cite{zdnet_2011:microsoft_contributes_linux}, Microsoft se unió a la Linux Foundation.\cite{arstechnica_2016:microsoft_joins_linux_foundation}

Poco después de haber comprado Xamarin, una herramienta para desarrollar apps multiplataforma, Microsoft anunció que habían liberado la herramienta y contribuído a la .NET Foundation con los SDKs de Xamarin para Android, iOS y Mac bajo la licencia MIT. Xamarin pasó de tener un alto coste por licencia a ser gratuita y estar integrada en el stack de .NET. \cite{genbeta_2016:microsoft_xamarin} \cite{petri_2016:microsoft_xamarin}

En este mismo año se publicó el \href{https://docs.microsoft.com/es-es/windows/wsl/about}{Windows Subsystem for Linux (WSL)}\footnote{https://docs.microsoft.com/es-es/windows/wsl/about}, que permite a los desarrolladores ejecutar un entorno GNU/Linux, incluyendo la mayoría de herramientas de línea de comandos, utilidades y aplicaciones, directamente en Windows.

\subsubsection{2018: Compra de Github y Open Invention Network}
Microsoft se hizo en 2018 con Github por nada menos que 7500 millones de dólares. Github es hoy por hoy el mayor servicio de repositorios Git y el favorito entre los desarrolladores de proyectos Open Source en todo el mundo. Con su compra, Microsoft se acerca a los 27 millones de desarrolladores registrados en la plataforma. \cite{xataca_2018:microsoft_gighub}

También en 2018, Microsoft se unió a la Open Invention Network. Se trata de una comunidad para evitar agresiones en el ámbito de las patentes que apoya la libertad de acción en Linux como elemento clave del software Open Source. La OIN adquiere patentes y las licencia sin coste a los miembros de su comunidad. A cambio, éstos acuerdan no utilizar sus propias patentes contra Linux y las aplicaciones y sistemas relacionados\cite{wiki_2020:open_invention_network}. Al unirse a la OIN, Microsoft abrió 60000 patentes al grupo\cite{azure_blogs_2018:microsoft_oin}.

\subsubsection{2019: Charla de Richard Stallman en Microsoft}
Richard Stallman fue invitado a dar una charla en Redmond el 4 de septiembre de 2019. Según el propio Stallman, algunos ejecutivos de Microsoft están seriamente interesados en los asuntos éticos que rodean al software. En su charla, Stallman presentó una serie de sugerencias que ayudarían a la comunidad de software libre. En un artículo sobre la conferencia, Stallman dice que <<Deberíamos juzgar los actos futuros de Microsoft por su naturaleza y sus efectos. Sería un error juzgar una acción más duramente si la hace Microsoft que si la hace cualquier otra compañía.>> <<Debemos juzgar a Microsoft en el futuro por lo que haga entonces.>> \cite{stallman_2019}

\subsection{Las motivaciones del cambio}
¿Qué ha motivado el cambio radical de postura de Microsoft? Como es normal en el mundo empresarial, el motor del cambio es el dinero. Con la llegada del cloud computing y ese es un territorio del software libre. En este sentido se podría decir que no le ha quedado más remedio que ofrecer plataformas abiertas en Azure como linux, MySQL, Docker, etc. Pero Microsoft ha hecho más que eso. Con sus movimientos desde 2014, Microsoft ha conseguido expandir el uso de sus tecnologías y ganarse la confianza y el respeto de los desarrolladores.