\section{El estilo Microsoft}
Microsoft es una multinacional fundada el 4 de abril de 1975 por Bill Gates y Paul Allen en Albuquerque, Nuevo México. Desde sus inicios, la compañía se ha caracterizado por su cuestionable ética y su agresiva estrategia empresarial, así como por el uso de su posición dominante para <<exterminar>> a la competencia e imponer sus productos y sus estándares. Prueba de ello son los \href{https://en.wikipedia.org/wiki/Microsoft_litigation}{numerosos juicios} en los que la compañía se ha visto envuelta\cite{wiki_2020:microsoft_litigation}. Con el fin de formar una imagen de la compañía, se resumen a continuación algunos hechos que tuvieron a Microsoft como protagonista.

\subsection{IBM y Microsoft}
A principios de los 80 IBM dominaba cómodamente el mercado de la computación para empresas. Con la llegada de los ordenadores personales, la compañía decidió contratar los servicios de Microsoft para el desarrollo de su sistema operativo para sus IBM-PC. Para ello, Microsoft compró los derechos no-exclusivos del sistema operativo QDOS, en el cuál basaría el desarrollo del nuevo sistema, a la empresa SCP. Microsoft desarrolló MS-DOS y lo licenció a IBM como PC-DOS. El único componente exclusivo para IBM era la BIOS.\cite{tdith_deal_with_devil_2019}\cite{tdith_rights_to_86DOS:2019}

Supuestamente, el apretado calendario del nuevo sistema operativo llevó a Microsoft a utilizar malas prácticas en el desarrollo. Esto tuvo como consecuencia un bajo rendimiento de las llamadas al sistema a través de las funciones estándar de la BIOS en comparación con los accesos al sistema saltándose la BIOS. Este hecho llevó a muchos desarrolladores de videojuegos (por ejemplo, Microsoft Flight Simulator 1.0, 1982) a ignorar la BIOS, lo que revelaba la arquitectura del sistema en la programación, permitiendo a otros fabricantes de hardware obtener pistas sobre la arquitectura del IBM-PC rápidamente con el fin de clonarla. Ya en 1982 Compaq lanzó el Compaq Portable, el primer ordenador PC-Compatible. La rápida expansión del mercado de PCs IBM-Compatibes llevó a Microsoft, que inteligentemente había incluído en su contrato con IBM una cláusula que les permitía vender su sistema operativo MS-DOS a clientes diferentes de IBM, a convertirse en la compañía líder en software de ordenadores personales, desplazando a IBM de su hegemonía total.\cite{wiki:ibm-pc-compatible-origins}

\subsection{La guerra de los navegadores}
Durante los años 90 tuvo lugar la conocida como <<\href{https://es.wikipedia.org/wiki/guerra_de_navegadores}{Guerra de Navegadores}>>\cite{wiki:guerra_navegadores_2020} en la que Microsoft logró imponer su producto Internet Explorer distribuyéndolo gratuitamente junto con Windows 95, lo que asfixió comercialmente a su principal competidor, Netscape Communicator, que contaba con un producto mejor. Aunque Microsoft fue condenado en un \href{https://es.wikipedia.org/wiki/Caso_Estados_Unidos_contra_Microsoft}{juicio por monopolio}\cite{wiki:juicio_microsoft_2020}, tras las apelaciones y el acuerdo final, las medidas adoptadas finalmente no obligaban a la compañía a realizar cambios efectivos en su estructura o su política comercial.\cite{thotw_2019:Browser_Wars}

Años después tuvo lugar un \href{https://en.wikipedia.org/wiki/Microsoft_Corp._v._Commission}{juicio similar en la Unión Europea}\cite{wiki_2020:microsoft_v_commision} por abuso de posición dominante en mercado a raíz de protestas de Novell y SUN Microsystems. Este juicio acabó obligando a Microsoft a que divulgara información sobre algunos de sus productos y liberara una versión de Windows sin Windows Media Player

\subsubsection{SCO vs IBM}
En 2003, el grupo SCO (antes Caldera International) afirmó ser el dueño de Unix y de otros sistemas operativos, y que Linux y otras variantes de Unix estaban violando su propiedad intelectual porque usaban código de Unix sin licencia. Primero SCO intentó estructurar un proceso directamente contra los usuarios finales y las compañías proveedoras, lo que hubiera extendido en el mercado el miedo a usar estos sistemas. La respuesta fueron varias demandas de copyright contra SCO, opr ejemplo por vulnerar la GPL. Al final, SCO emprendió sólo unos cuantos procesos legales contra IBM, Novell, DaimlerChrysler y Autozone. A su vez, RedHat demandó a SCO por realizar falsas afirmaciones que afectaban a sus negocios. \cite{wiki_2019:sco_linux}

Esta campaña de SCO fue sustancialmente financiada y promovida por Microsoft y por negocios fondos de inversión fuertemente vinculados con Redmond. En particular, un email interno filtrado de SCO indicaba que Microsoft había invertido 106 millones de dólares a través de BayStar y otros medios. El documento fue reconocido como verídico por Blake Stowell de SCO. BayStar afirmó que el acuerdo fue propuesto por Microsoft, pero que el dinero vino de otras fuentes. Además, en mayo de 2003, Microsoft pagó a SCO 6 millones de dólares por una licencia de <<Unix and Unix-related patents>>, a pesar de que SCO no poseía ninguna patente así. Este acuerdo fue reflejado en la prensa como apoyo financiero de Microsoft a SCO en su disputa legal contra IBM por, supuestamente, violar la licencia de Unix en el desarrollo de Linux.\cite{wiki_2019:sco_linux_microsoft_controverse}

En 2016, la mayoría de los casos habían sido resueltos y ninguna sentencia fue favorable a SCO.\cite{wiki_2019:sco_linux}

\subsection{Embrace, extend and extinguish}
El departamento de justicia de los estados unidos descubrió que Microsoft usaba la expresión <<\href{https://en.wikipedia.org/wiki/Embrace,_extend,_and_extinguish}{embrace, extend and extinguish}>> internamente para describir su estrategia por la cuál Microsoft decide en un primer momento <<adoptar>> un estándar público, después lo <<mejora>> añadiendo extensiones incompatibles con el estándar original, y termina <<extinguiendo>> el estándar público al imponer sus propias extensiones propietarias por medio de su dominio del mercado.\cite{wiki_2020:embrace_extend_extinguish}

\label{par:FUD}
Otra estrategia de dudosa ética utilizada por Microsoft fue la llamada <<\href{https://en.wikipedia.org/wiki/Fear,_uncertainty,_and_doubt}{FUD}>> por sus siglas en inglés <<Fear, Uncertainty and Doubt>>\cite{wiki_2020:FUD}. En el primero de los \href{https://en.wikipedia.org/wiki/Halloween_documents}{Halloween Documents}\cite{wiki_2020:halloween_documents}, se revela que Microsoft utilizaba normalmente dicho método para desacreditar públicamente los productos de la competencia.